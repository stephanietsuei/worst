\documentclass{article}

\usepackage{fullpage}
\usepackage{amsmath}
\usepackage{graphicx}
\usepackage{bm}
\usepackage{subfigure}

\newcommand{\xb}{\mathbf{x}}
\newcommand{\ub}{\mathbf{u}}
\newcommand{\vb}{\mathbf{v}}
\newcommand{\zb}{\mathbf{z}}
\newcommand{\nub}{\bm{\nu}}
\newcommand{\xib}{\bm{\xib}}
\newcommand{\gammab}{\bm{\gamma}}
\newcommand{\kappab}{\bm{\kappa}}
\newcommand{\lambdab}{\bm{\lambda}}
\newcommand{\deltab}{\bm{\delta}}
\newcommand{\Ub}{\mathbf{U}}
\newcommand{\Yb}{\mathbf{Y}}
\newcommand{\hb}{\mathbf{h}}
\newcommand{\yb}{\mathbf{y}}


\author{Stephanie Tsuei}
\title{The Worst Toolbox Documentation}
\date{Version 0.1a}

\begin{document}
\maketitle

\section{Introduction}
This toolbox implements the algorithm described in:

\begin{quote}
J. E. Tierno, R. M. Murray, J. C. Doyle, and I. M. Gregory,
``Numerically Efficient Robustness Analysis of Trajectory Tracking for Nonlinear
Systems," \emph{J. Guid. Control. Dyn.}, vol. 20, no. 4, pp. 640-647, Jul.
1997.
\end{quote}

which uses theory presented in:

\begin{quote}
A. Bryson and Y. Ho, \emph{Applied optimal control: optimization,
estimation, and control}, vol. 59, no. 8. 1975.
\end{quote}



\subsection{The Robust Trajectory Tracking Problem}

\begin{figure}
\begin{center}
\includegraphics[width=4in]{system}
\caption{A block diagram dipicting the robust trajectory generation problem.}
\end{center}
\label{sys}
\end{figure}

Consider the dynamical system pictured in Figure \ref{sys} with dynamics
\[ \dot{\xb}(t) = f(\xb(t),\Ub(t),\ub(t),\vb(t),\deltab) \]
and outputs
\begin{align*}
\Yb(t) &= g(\xb(t),\Ub(t),\ub(t),\vb(t),\deltab) \\
\zb(t) &= h(\xb(t),\Ub(t),\ub(t),\vb(t),\deltab)
\end{align*}
where $\xb(t)$ is the state, $\Ub(t)$ is a nominal input signal, $\ub(t)$ is a
disturbance signal, $\delta$ is a vector of uncertain parameters, $\Delta$ is
an uncertain block of unit norm, and $\vb(t)$ and $\zb(t)$ are signals
representing any possible unmodeled dynamics. Therefore, the nominal
trajectory, is given by
\[ \Yb_n(t) = g(\xb(t),\Ub(t),0,0,\deltab_n) \]
where $\deltab_n$ are the nominal values of the uncertain parameters and the
quantity $\yb(t) = \Yb(t) - \Yb_n(t)$ is the error signal.


We make the following assumptions:
\begin{itemize}
	\item $\|\vb(t)\| = \|\zb(t)\|$
	\item $\vb(t)$ and $\zb(t)$ can be written in block form $\vb(t) =
	      \begin{bmatrix} \vb_1(t) & \vb_2(t) & \dots & \vb_p(t) 
		  \end{bmatrix}^T$
		  and $\zb(t) = \begin{bmatrix} \zb_1(t) & \zb_2(t) & \dots & \zb_p(t)
		  \end{bmatrix}^T$. The total number of blocks in $\vb(t)$
		  and $\zb(t)$ must be the same, but $\vb_i(t)$ and $\zb_i(t)$ can be of
		  different dimensions.
    \item $\|\ub(t)\|$ is a known constant scalar $M$ if there is only one
    	  disturbance input. If there are multiple disturbance inputs
		  $\ub_1(t)$, $\ub_2(t)$, then each disturbance input $i$ has a known 
		  norm $M_i$.
	\item $\underline{\deltab} \le \deltab_n \le \bar{\deltab}$ (elementwise) 
	      where $\underline{\deltab}$ and $\bar{\deltab}$ are known constants.
	\item We are only interested in the finite time interval $(t_i, t_f)$ and 
	      for a vector-valued time signal $\mathbf{w}(t)$, \[\|\mathbf{w}(t)\|  
		  = \left (\frac{1}{t_f-t_i} \int_{t_i}^{t_f}\mathbf{w}^T(t) 
		  \mathbf{w}(t) dt \right )^{\frac{1}{2}}\]
\end{itemize}

Given the information above, this toolbox computes the functions $\ub(t)$,
$\vb(t)$ and the value of $\deltab(t)$ that \emph{maximizes} the quantity
below: 
\[ J =  \|\yb(t)\| = \left ( \frac{1}{t_f - t_i} \int_{t_i}^{t_f} \yb(t)^T
\yb(t) dt \right )^{\frac{1}{2}}  \]


\subsection{System Requirements}
Requirements:
\begin{itemize}
\item Matlab R2012b or later
\item Simulink
\end{itemize}
The toolbox may work on some earlier versions of Matlab and Simulink, but was
developed on Matlab R2012b.


\subsection{Install Instructions}
To install the Worst Toolbox, place the folder \texttt{worst/} anywhere on your
local machine and add it (but not its subdirectories) to the Matlab path. This
is equivalent to the running the command:

\begin{quote}\texttt{addpath(`/path/to/robust/worst')} \end{quote}

\section{Using Worst}

\subsection{Simulink Model Configuration}
The Worst toolbox makes some assumptions about the configuration of Simulink
models:

\subsubsection*{Order of Inputs}

The program assumes that models have up to four input signals and two output
signals. Each of the signals may have arbitrary dimension to accomodate multiple
inputs, disturbances, and feedback signals. The order assumed is:
\begin{enumerate}
\item Nominal inputs, $\Ub(t)$
\item Disturbance inputs, $\ub(t)$
\item Unmodeled feedback inputs, $\vb(t)$
\item Uncertain Parameters, $\deltab$
\end{enumerate}

\subsubsection*{Order of Outputs}

Models are allowed to have an arbitrary number of outputs. If the Simulink model
has a total of $r$ outputs, and the first $m <= r$ outputs are outputs included
in computing the performance measure $J$, then the remaining $r - m$ outputs are
unmodeled feedback outputs; the coordinates of $z$.


\subsection{Syntax of \texttt{worst}}




\subsection{Speeding up the Program}

\begin{figure}
\begin{center}
\subfigure[Original Simulink
Diagram]{\includegraphics[width=3.2in]{dubin\string_diagram}} 
\subfigure[Diagram
with S-Function]{\includegraphics[width=3.2in]{dubin\string_diagram\string_sfun}}
\caption{A model of Dubin's Car compiled into an S-function.}
\end{center}
\label{sfunc}
\end{figure}

The program repeatedly calls \texttt{linmod} in order to build and simulate an
adjoint system. This greatly slows down the program because Simulink will
recompile the model with every call. To hasten computation, compile your
Simulink model into an S-function before running the program, as shown in Figure
\ref{sfunc}. The more blocks in the original Simulink model, the greater the
improvement in speed.


\section{Examples}
\subsection{A Linear System}
\subsection{Dubin's Car}
\subsection{The Caltech Ducted Fan}



\end{document}
